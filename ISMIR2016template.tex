\documentclass{article}
\usepackage{ismir,amsmath,cite}
\usepackage{graphicx}
\usepackage{color}

\title{Extracting Ground-Truth Information from MIDI Files}

%\oneauthor
% {Colin Raffel and Daniel P. W. Ellis}
% {LabROSA \\ Department of Electrical Engineering \\ Columbia University \\ New York, NY}

\threeauthors
  {First Author} {Affiliation1 \\ {\tt author1@ismir.edu}}
  {Second Author} {\bf Retain these fake authors in\\\bf submission to preserve the formatting}
  {Third Author} {Affiliation3 \\ {\tt author3@ismir.edu}}

%% To make customize author list in Creative Common license, uncomment and customize the next line
%  \def\authorname{Colin Raffel, Daniel P. W. Ellis} 

\sloppy

\begin{document}

\maketitle

\begin{abstract}
Abstract
\end{abstract}

\section{MIDI Files}\label{sec:introduction}

What is MIDI?

What are MIDI files?

Where did they all come from? (Small filesize, hobbyist musicians) ringtones, karaoke

Overview of paper

\section{Information Available in MIDI Files}

\subsection{Transcription}

For transcription and score-informed source separation

Better chroma for chord

Instrument activation

\subsection{Onsets}

Thanks to transcription.  However, of limited utility.

\subsection{Musicological Features}

Reference jSymbolic, music21

\subsection{Meter}

Beats and downbeats via time signature and tempo change events

\subsection{Key}

Key change events

\section{Utilizing MIDI Files as Ground Truth}

\subsection{Matching}

Required for nearly everything.

For small collections, you might not have to do anything.

For large collections, you need sophisticated methods because of lack of metadata.

Beneficial for MIDI analysis too, thanks to matching it with audio metadata collections.

\subsection{Aligning}

Required for all time-sensitive information.

\subsection{Evaluating Quality}

In some cases, the transcription may be bad.

\subsection{Extracting Information}

Short description of \texttt{pretty\_midi}.

\section{Measuring a Baseline of Reliability for MIDI-Derived Ground Truth}

Match and align to Isophonics Beatles collection

\subsection{Beat Experiment}

\subsection{Key Experiment}

\section{Discussion}

As-is, a lot of the information is usable.

Improving audio-to-MIDI alignment as a proxy for getting additional metadata.

Many additional avenues for improvement of measuring quality.

\end{document}
